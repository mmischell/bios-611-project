\documentclass[11pt]{article}
\usepackage[utf8]{inputenc}
\usepackage[T1]{fontenc}
\usepackage{graphicx}
\usepackage{wrapfig}
\usepackage{rotating}
\usepackage{amsmath}
\usepackage{amssymb}
\usepackage{hyperref}

\title{Exploration of Obesity and Risk Factors}
\author{Melissa Mischell}
\date{\today}

\begin{document}
\maketitle

\begin{figure}[h]
\includegraphics[width=.9\linewidth]{./figures/national_gender_plt.png}
\caption{National obesity rate by gender}
\label{fig:gender}
\end{figure}

\begin{figure}[h]
\includegraphics[width=.9\linewidth]{./figures/nc_phys_legislation.png}
\caption{
  Percent of people in NC who engage in no leisure-time physical activity. 
  Red lines indicate NC legislation related to physical activity enacted.
}
\label{fig:phys_act}
\end{figure}

\begin{figure}[h]
\includegraphics[width=.9\linewidth]{./figures/nc_obesity_leg.png}
\caption{
  Percent of people in NC with obesity and NC legislation enacted related to obesity and risk factors. 
}
\label{fig:nc_obesity}
\end{figure}

\begin{figure}[h]
\includegraphics[width=.9\linewidth]{./figures/perc_obesity_pca.png}
\caption{
  Percent of people with obesity and most significant component from linear model and PCA with formula percent obesity ~ pca components derived from stratifications   
}
\label{fig:nc_obesity}
\end{figure}

\begin{figure}[h]
\includegraphics[width=.9\linewidth]{./figures/perc_obesity_fruit.png}
\caption{
  Percent of people with obesity against percent that consumer fruit less than one time daily
}
\label{fig:obesity_fruit}
\end{figure}

\end{document}
