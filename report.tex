\documentclass[11pt]{article}
\usepackage[utf8]{inputenc}
\usepackage[T1]{fontenc}
\usepackage{graphicx}
\usepackage{wrapfig}
\usepackage{rotating}
\usepackage{amsmath}
\usepackage{amssymb}
\usepackage{hyperref}

\title{Exploration of Obesity and its Risk Factors}
\author{Melissa Mischell}
\date{\today}

\begin{document}
\maketitle

\begin{figure}[hp]
\includegraphics[width=.45\linewidth]{./figures/national_gender_plt.png}
\includegraphics[width=.45\linewidth]{./figures/income_time_plt.png}
\includegraphics[width=.45\linewidth]{./figures/race_time_plt.png}
\includegraphics[width=.45\linewidth]{./figures/edu_time_plt.png}
\includegraphics[width=.45\linewidth]{./figures/age_time_plt.png}
\caption{
  National obesity rates by gender, race, and education level over time. There appears to be a recent shift toward higher obesity rates for women, Hawaiian/Pacific Islanders, and people with less than high school level of education. 
}
\label{fig:dems_by_time}
\end{figure}

\begin{figure}[h]
\includegraphics[width=.9\linewidth]{./figures/income_phys_plt.png}
\caption{
  Percent of people in each weight classification and with no leisure physical activity by income in 2020.
}
\label{fig:income_pa}
\end{figure}


\begin{figure}[h]
\includegraphics[width=.9\linewidth]{./figures/nc_phys_legislation.png}
\caption{
  Percent of people in NC who engage in no leisure-time physical activity. 
  Red lines indicate NC legislation related to physical activity enacted.
}
\label{fig:phys_act}
\end{figure}

\begin{figure}[h]
\includegraphics[width=.9\linewidth]{./figures/nc_obesity_leg.png}
\caption{
  Percent of people in NC with obesity and NC legislation enacted related to obesity and risk factors. 
}
\label{fig:nc_obesity}
\end{figure}


\begin{figure}[h]
\includegraphics[width=.9\linewidth]{./figures/perc_obesity_fruit.png}
\caption{
  Percent of people with obesity against percent that consume fruit less than one time daily
}
\label{fig:obesity_fruit}
\end{figure}


\end{document}
